\documentclass[conference]{IEEEtran}
\IEEEoverridecommandlockouts
% The preceding line is only needed to identify funding in the first footnote. If that is unneeded, please comment it out.
\usepackage{cite}
\usepackage{amsmath,amssymb,amsfonts}
\usepackage{algorithmic}
\usepackage{graphicx}
\usepackage{textcomp}
\usepackage{xcolor}
\usepackage{placeins}
\def\BibTeX{{\rm B\kern-.05em{\sc i\kern-.025em b}\kern-.08em
    T\kern-.1667em\lower.7ex\hbox{E}\kern-.125emX}}
\begin{document}

\title{Rangefinder-Calibrated Visual Object Detection}

\author{
	\IEEEauthorblockN{Raj Anadkat}
	\IEEEauthorblockA{raj618@seas.upenn.edu}
\and
	\IEEEauthorblockN{Jonathan Lee}
	\IEEEauthorblockA{jonlee27@seas.upenn.edu}
\and
	\IEEEauthorblockN{Jonathan Schoeffling}
	\IEEEauthorblockA{jschoeff@seas.upenn.edu}
\and
	\IEEEauthorblockN{Hanli Zhang}
	\IEEEauthorblockA{hanlizh@seas.upenn.edu}
}

\maketitle

\begin{abstract}
The lidar used on the F1TENTH platform is expensive, which limits the size of
the classes and programs that use it. This project aimed to achieve a more
useful combined estimate of both azimuth and range by fusing data from high
azimuth accuracy / low range accuracy sensors with data from low azimuth
accuracy / high range accuracy accuracy sensors. Additional work is needed to
overcome noise in the depth estimation model and to reliably associate the
measurements of each sensor.
\end{abstract}

\begin{IEEEkeywords}
F1TENTH, autonomous driving, MiDaS, time-of-flight, rangefinder, lidar, sensor
fusion
\end{IEEEkeywords}

\section{Approach}

\section{Implementation}
\subsection{MiDaS Depth Inference Model}
\subsection{Time-of-Fight Array}
\subsection{Sensor Fusion}
\subsection{Gap Following}

\section{Results}
\subsection{Integrated Gap Following}
\subsection{MiDaS Performance}
\subsection{Time-of-Fight Array Performance}
\subsection{Sensor Fusion Performance}

\section{Future Work}



\begin{thebibliography}{00}
\bibitem{midas} https://github.com/isl-org/MiDaS
\bibitem{midas-paper} Ranftl et. al., Towards Robust Monocular Depth Estimation:
Mixing Datasets for Zero-shot Cross-dataset Transfer, TPAMI 2022
\bibitem{realsense-range} https://www.intelrealsense.com/stereo-depth/
\bibitem{realsense-error} https://dev.intelrealsense.com/docs/tuning-depth-cameras-for-best-performance
\end{thebibliography}

\end{document}
